% Options for packages loaded elsewhere
\PassOptionsToPackage{unicode}{hyperref}
\PassOptionsToPackage{hyphens}{url}
%
\documentclass[]{article}
\usepackage{amsmath,amssymb}

\usepackage{booktabs}
\usepackage{subcaption}
\usepackage{iftex}
\ifPDFTeX
  \usepackage[T1]{fontenc}
  \usepackage[utf8]{inputenc}
  \usepackage{textcomp} % provide euro and other symbols
\else % if luatex or xetex
  \usepackage{unicode-math} % this also loads fontspec
  \defaultfontfeatures{Scale=MatchLowercase}
  \defaultfontfeatures[\rmfamily]{Ligatures=TeX,Scale=1}
\fi
\usepackage{lmodern}
\ifPDFTeX\else
  % xetex/luatex font selection
\fi
% Use upquote if available, for straight quotes in verbatim environments
\IfFileExists{upquote.sty}{\usepackage{upquote}}{}
\IfFileExists{microtype.sty}{% use microtype if available
  \usepackage[]{microtype}
  \UseMicrotypeSet[protrusion]{basicmath} % disable protrusion for tt fonts
}{}
\makeatletter
\@ifundefined{KOMAClassName}{% if non-KOMA class
  \IfFileExists{parskip.sty}{%
    \usepackage{parskip}
  }{% else
    \setlength{\parindent}{0pt}
    \setlength{\parskip}{6pt plus 2pt minus 1pt}}
}{% if KOMA class
  \KOMAoptions{parskip=half}}
\makeatother
\usepackage{xcolor}
\setlength{\emergencystretch}{3em} % prevent overfull lines
\providecommand{\tightlist}{%
  \setlength{
\itemsep}{0pt}\setlength{\parskip}{0pt}}
\setcounter{secnumdepth}{-\maxdimen} % remove section numbering
\ifLuaTeX
  \usepackage{selnolig}  % disable illegal ligatures
\fi
\usepackage{bookmark}
\IfFileExists{xurl.sty}{\usepackage{xurl}}{} % add URL line breaks if available

\usepackage{tikz}
\usepackage{pgf}
\usetikzlibrary{
  arrows.meta,
  decorations.pathmorphing,
  backgrounds,positioning,
  fit,
  petri,
  graphs,
  graphdrawing
} 
\usegdlibrary{circular}

\begin{document}
% \resizebox{200}{!}
% {
% \begin{tikzpicture}
% 	[
% 		node distance=0mm,
% 		box/.style={
% 				rectangle,
% 				minimum size=6mm,
% 				ultra thin,
% 				draw=black!40,
% 				fill=black!10,
% 			},
% 		emptyBox/.style={
% 				rectangle,
% 				minimum height=18mm,
% 				minimum width=6mm,
% 				ultra thin,
% 				draw=black!40,
% 				fill=white,
% 			},
% 	]
% 	\useasboundingbox (-5,-5) rectangle (5,5);
% 	\begin{scope}[transform canvas={scale=10}]
% 	\matrix[row sep=0mm,column sep=0mm, every node/.style=box] {
% 		\node (n6) {}; & \node (n7) {}; & \node (n8)  {}; \\
% 		\node (n5) {}; &                & \node (n9)  {}; \\
% 		\node (n4) {}; &                & \node (n10) {}; \\
% 		\node (n3) {}; &                & \node (n11) {}; \\
% 		\node (n2) {}; & \node (n1) {}; & \node (n12) {}; \\
% 	};
% 	\node [emptyBox, below= of n7 ] {};

% 	\begin{scope}[font=\sffamily]
% 		\node at (n1)  [] {D};
% 		\node at (n2)  [] {};
% 		\node at (n3)  [] {};
% 		\node at (n4)  [] {F};
% 		\node at (n5)  [] {};
% 		\node at (n6)  [red] {G};
% 		\node at (n7)  [] {};
% 		\node at (n8)  [fill=red] {A};
% 		\node at (n9)  [] {};
% 		\node at (n10) [fill=green] {B};
% 		\node at (n11) [] {C};
% 		\node at (n12) [] {};
% 		\draw [->, green, ultra thick] (n12) to [bend left] (n8);
% 		\draw [->, red, ultra thick, bend right] (n4) -- (n11);
% 	\end{scope}
% 	\end{scope}
% \end{tikzpicture}

\begin{figure}
	\begin{subfigure}{.30\textwidth}
	  \centering
		\begin{tikzpicture}
		  every label/.style={green},
		  radius=1.25cm,
		  ]
		  \graph [    
			simple necklace layout,
			node sep=0pt, 
			grow'=up,
			node distance=0pt,    
			nodes={
			  draw,
			  circle, 
			  as =  , 
			  minimum size=20pt, 
			  font=\sffamily 
			}
		  ]
		  { 
			   1  [as = D] 
			-- 2  [draw=none]
			-- 3  [as = E]
			-- 4  [as = F]
			-- 5  [draw=none]
			-- 6  [as = G]
			-- 7  [draw=none]
			-- 8  [as = A]
			-- 9  [draw=none]
			-- 10 [as = B]
			-- 11 [as = C]
			-- 12 [draw=none]
		  };
		  \draw [] (0,39.50pt) circle [radius=28pt];
		  \draw[ultra thick, color=red] (4) -- (10);
		\end{tikzpicture}
	  \caption{\sffamily{WHITE} con su centro tritonal en rojo}
	  \label{fig:tikz1}
	\end{subfigure}
	\hfill
	\begin{subfigure}{0.30\textwidth}
	  \centering
		\begin{tikzpicture}
		  every label/.style={green},
		  radius=1.25cm,
		  ]
		  \graph [    
			simple necklace layout,
			node sep=0pt, 
			grow'=up,
			node distance=0pt,    
			nodes={
			  draw,
			  circle, 
			  as =  , 
			  minimum size=20pt, 
			  font=\sffamily 
			}
		  ]
		  { 
			   1  [as = D] 
			-- 2  [draw=none]
			-- 3  [as = E]
			-- 4  [as = F]
			-- 5  [draw=none]
			-- 6  [as = G]
			-- 7  [draw=none]
			-- 8  [as = A]
			-- 9  [draw=none]
			-- 10 [as = B]
			-- 11 [draw=none]
			-- 12 [as = d]
		  };
		  \draw [] (0,39.50pt) circle [radius=28pt];
		  \draw[ultra thick, color=red] (4) -- (10);
		  \draw[ultra thick, color=red] (6) -- (12);
		\end{tikzpicture}
	  \caption{Agregamos un tritono manteniendo el anterior. Ahora estamos en \sffamily{BLUE}}
	  \label{fig:tikz1}
	\end{subfigure}
	\hfill
	\begin{subfigure}{0.30\textwidth}
	  \centering
		\begin{tikzpicture}
		  every label/.style={green},
		  radius=1.25cm,
		  ]
		  \graph [    
			simple necklace layout,
			node sep=0pt, 
			grow'=up,
			node distance=0pt,    
			nodes={
			  draw,
			  circle, 
			  as =  , 
			  minimum size=20pt, 
			  font=\sffamily 
			}
		  ]
		  { 
			   1  [as = A] 
			-- 2  [draw=none]
			-- 3  [as = B]
			-- 4  [draw=none]
			-- 5  [as = d]
			-- 6  [as = D]
			-- 7  [draw=none]
			-- 8  [as = E]
			-- 9  [as = F]
			-- 10 [draw=none]
			-- 11 [as = G]
			-- 12 [draw=none]
		  };
		  \draw [] (0,39.50pt) circle [radius=28pt];
		  \draw[ultra thick, color=red] (3) -- (9);
		  \draw[ultra thick, color=red] (5) -- (11);
		\end{tikzpicture}
	  \caption{\sffamily{BLUE} con su segundo centro tritonal, reorientada por el oyente para sentir el nuevo centro tonal: \sffamily{A}}
	  \label{fig:tikz1}
	\end{subfigure}
	\hfill
	\begin{subfigure}{0.30\textwidth}
		\centering
		  \begin{tikzpicture}
			every label/.style={green},
			radius=1.25cm,
			]
			\graph [    
			  simple necklace layout,
			  node sep=0pt, 
			  grow'=up,
			  node distance=0pt,    
			  nodes={
				draw,
				circle, 
				as =  , 
				minimum size=20pt, 
				font=\sffamily 
			  }
			]
			{ 
				 1  [draw=none] 
			  -- 2  [as = b]
			  -- 3  [as = B]
			  -- 4  [draw=none]
			  -- 5  [as = d]
			  -- 6  [draw= none]
			  -- 7  [as= e]
			  -- 8  [draw=none]
			  -- 9  [as = F]
			  -- 10 [draw=none]
			  -- 11 [as = G]
			  -- 12 [as=a]
			};
			\draw [] (0,39.50pt) circle [radius=28pt];
			\draw[ultra thick, color=red] (3) -- (9);
			\draw[ultra thick, color=red] (5) -- (11);
		  \end{tikzpicture}
		\caption{\sffamily{BLUE} cambiamos notas para invertir la estructura cambiando a un nuevo centro tonal: \sffamily{e}}
		\label{fig:tikz1}
	  \end{subfigure}
	  \hfill	
	  \begin{subfigure}{0.30\textwidth}
		\centering
		  \begin{tikzpicture}
			every label/.style={green},
			radius=1.25cm,
			]
			\graph [    
			  simple necklace layout,
			  node sep=0pt, 
			  grow'=up,
			  node distance=0pt,    
			  nodes={
				draw,
				circle, 
				as =  , 
				minimum size=20pt, 
				font=\sffamily 
			  }
			]
			{ 
				 1  [as= e] 
			  -- 2  [draw=none]
			  -- 3  [as= F]
			  -- 4  [draw=none]
			  -- 5  [as = G]
			  -- 6  [as = a]
			  -- 7  [draw=none]
			  -- 8  [as = b]
			  -- 9  [as = B]
			  -- 10 [draw=none]
			  -- 11 [as = e]
			  -- 12 [draw=none]
			};
			\draw [] (0,39.50pt) circle [radius=28pt];
			\draw[ultra thick, color=red] (3) -- (9);
			\draw[ultra thick, color=red] (5) -- (11);
		  \end{tikzpicture}
		\caption{\sffamily{BLUE} Invertimos de nuevo para volver a la estructura estandar}
		\label{fig:tikz1}
	  \end{subfigure}
	  \hfill	
	\label{fig:modulation-changing-tonal-center}
	\caption{La base de la sustitución tritonal}
	\end{figure}

\end{document}